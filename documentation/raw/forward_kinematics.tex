\documentclass{article}

\usepackage{cmap}  % should be before fontenc
\usepackage[T2A]{fontenc}
\usepackage[utf8]{inputenc}
\usepackage[russian]{babel}
\usepackage{amsmath,amssymb,amsthm}
\usepackage{verbatim}
\usepackage[pdftex,colorlinks=true,linkcolor=blue,urlcolor=red,unicode=true,hyperfootnotes=false,bookmarksnumbered]{hyperref}
\usepackage{indentfirst}

\begin{document}

\section{Прямая задача}

Конфигурация описывается набором параметров переходов от $i$-й СК к $i+1$-й ($i$ от 1 до количества джоинтов):
- $d_i$ - смещение вдоль $z_i$
- $\theta_i$ - поворот вокруг $z_i$, после которого $x_i$ стало бы параллельной $x_{i+1}$
- $\alpha_i$ - поворот вокруг $x_{i+1}$, после которого $z_i$ стало бы параллельной $z_{i+1}$
- $a_i$ - смещение вдоль $x_{i+1}$

Поворотные joint-ы тоже вращаются вокруг оси $z_i$, потому между $\theta_i$ у углами поворотов joint-ов ($j_i$) есть связь. Например в fanuc-е: 
- $j_1, j_4, j_5, j_6$ переходят в $\theta_1, \theta_4,\theta_5,\theta_6$ без изменений
- у $j_2$ направление вращение противоположно $z_2$, а также начальный угол смещен на $\frac{\pi}{2}\Rightarrow \theta_2=-j + \frac{\pi}{2}$
- $j_3$ показывает угол наклона относительно земли, что достигается связью $j_3$ с $j_2\Rightarrow \theta_3=j_3+j_2$

Однородная матрица перехода - произведение 4 однородных матриц: $$A_i = R_{z, \theta_i}\text{Trans}_{z, d_i}\text{Trans}_{z, a_i}R_{x,\alpha_i}=\begin{bmatrix}
c_{\theta_i} & -s_{\theta_i} c_{\alpha_i} & s_{\theta_i}s_{\alpha_i} & a_ic_{\theta_i}\\
s_{\theta_i} & c_{\theta_i} c_{\alpha_i} & -c_{\theta_i}s_{\alpha_i} & a_is_{\theta_i}\\
0 & s_{\alpha_i} & c_{\alpha_i} & d_i \\
0 & 0 & 0 & 1
\end{bmatrix}$$
$s_{\theta}=\sin(\theta), c_{\theta}=\cos(\theta)$, то же с $\alpha$ (подробнее - \href{https://uynitsuj.github.io/articles/2020-03/forward-kinematics-denavit-hartenberg-convention-on-a-6r-robotic-arm-example}{здесь})

Матрица, описывающая положение TCP: $H = A_1\cdots A_n$ - тоже однородная, т.е. имеет вид:
$$
\begin{bmatrix}
\text{\Large$R$}&\begin{matrix} x \\ y \\ z \end{matrix}\\
\begin{matrix} 0 & 0 & 0 \end{matrix} & 1
\end{bmatrix}
$$

$R$ - матрица поворота, $x, y, z$ - вектор смещения. Из $R$ вытаскиваются фиксированные углы $w, p, r$ (yaW, Pitch, Roll; \href{https://youtu.be/Ft7L5EsF_Kw?t=479}{видео про WPR в Fanuc}) по формулам:

$p=arctan2(-r_{31}, \sqrt{r_{11}^2+r_{21}^2})$

$w=arctan2(r_{32}/\cos(p), r_{33}/\cos(p)$

$r=arctan2(r_{21}/\cos(p), r_{11}/\cos(p)$

(\href{https://i.imgur.com/Bf728xV.png}{чуть подробнее о формулах})

\end{document}